\documentclass[t,usenames,dvipsnames]{beamer}
\usetheme{Copenhagen}
\setbeamertemplate{headline}{} % remove toc from headers
\beamertemplatenavigationsymbolsempty

\usepackage{amsmath, tikz, xcolor, array, graphicx}
\usetikzlibrary{arrows.meta}
\everymath{\displaystyle}

\title{Polar Form of Conics}
\author{}
\date{}

\AtBeginSection[]
{
  \begin{frame}
    \frametitle{Objectives}
    \tableofcontents[currentsection]
  \end{frame}
}

\begin{document}

\begin{frame}
    \titlepage
\end{frame}

\section{Analyze the graphs of conic sections in polar form.}

\begin{frame}{Intro}
Given a fixed line $L$, a point $F$ not on $L$, and a positive number $e$, a \alert{conic section} is the set of all points $P$ such that \vspace{10pt}
\[
\frac{\text{the distance from $P$ to $F$}}{\text{the distance from $P$ to $L$}} = e
\]
\pause
\newline\\

The line $L$ is called the \alert{directrix} of the conic section, the point $F$ is called a \alert{focus} of the conic section, and the constant $e$ is called the \alert{eccentricity} of the conic section.
\end{frame}


\begin{frame}{Eccentricity, Focus, and Directrix Line}
    The conic section has eccentricity $e$, a focus $F$ at the origin and directrix line $x = -d$: \newline\\

\begin{center}
    \begin{tikzpicture}
    \draw [->, >=stealth] (-3,0) -- (4,0) node [below, right] {$x$};
    \draw [->, >=stealth] (0,-1) -- (0,4) node [above, right] {$y$};
    \coordinate (P) at (50:3.5);
    \coordinate (O) at (0,0);
    \draw [fill=black] (P) circle (1pt) node [right] {$P(r, \theta)$};
    \node at (O) [below right] {$O=F$};
    \draw [dashed] (O) -- (P) node [midway, above left] {$r$};
    \draw [<->, >=stealth, color=red] (-2.25,4) -- (-2.25,-0.5) node [below] {$x=-d$};
    \draw [<->, >=stealth, dashed, color=blue] (P) -- (0,2.68) node [midway, above] {$r\cos\theta$};
    \draw [<->, >=stealth, dashed, color=red] (-2.25,2.68) -- (0,2.68) node [midway, above] {$d$};
    \draw [->, >=stealth] (0:1) arc (0:50:1) node [midway, right] {$\theta$}; 
    \node at (-2.25,3.75) [red,left] {$L$};
    \end{tikzpicture}
\end{center}
\end{frame}

\begin{frame}{General Equation}
From which we get   
\[
e = \frac{\text{the distance from $P$ to $F$}}{\text{the distance from $P$ to $L$}} = \frac{r}{d+r\cos\theta} = e
\]
\pause
So that $r = e(d + r\cos\theta)$, and solving for $r$ gives us  
\begin{align*}
    \onslide<3->{r &= ed +er\cos\theta} \\[8pt]
    \onslide<4->{r - er\cos\theta &= ed} \\[8pt]
    \onslide<5->{r(1-e\cos\theta) &= ed} \\[8pt]
    \onslide<6->{r &= \frac{ed}{1-e\cos\theta}}
\end{align*}
\end{frame}

\begin{frame}{Example 1}
Examine the graphs of each of the following for different values of $d$, but with $e = 1$.  \newline\\
(a) \quad $r = \frac{ed}{1+e\cos\theta}$    \pause    \quad $\longrightarrow \quad r = \frac{d}{1+\cos\theta}$ \newline\\ \pause
    \begin{itemize}
        \item Parabola (opens left or right)  \pause  \newline\\
        \item Vertex at $\left(\frac{1}{2}d, 0\right)$ \pause \newline\\
        \item $d > 0$ opens left    \pause  \newline\\
        \item $d < 0$ opens right   \pause  \newline\\
        \item $y$-intercepts at $\left(0, \pm d\right)$
    \end{itemize}
\end{frame}

\begin{frame}{Example 1}
(b) \quad $r = \frac{ed}{1-e\cos\theta}$        \pause    \quad $\longrightarrow \quad r = \frac{d}{1-\cos\theta}$ \newline\\ \pause
    \begin{itemize}
        \item Parabola (opens left or right)  \pause  \newline\\
        \item Vertex at $\left(-\frac{1}{2}d, 0\right)$ \pause \newline\\
        \item $d > 0$ opens right    \pause  \newline\\
        \item $d < 0$ opens left   \pause  \newline\\
        \item $y$-intercepts at $\left(0, \pm d\right)$
    \end{itemize}
\end{frame}

\begin{frame}{Example 1}
(c) \quad $r = \frac{ed}{1+e\sin\theta}$    \pause    \quad $\longrightarrow \quad r = \frac{d}{1+\sin\theta}$ \newline\\ \pause
    \begin{itemize}
        \item Parabola (opens up or down)  \pause  \newline\\
        \item Vertex at $\left(0, \frac{1}{2}d\right)$ \pause \newline\\
        \item $d > 0$ opens down    \pause  \newline\\
        \item $d < 0$ opens up   \pause  \newline\\
        \item $x$-intercepts at $\left(\pm d, 0\right)$
    \end{itemize}
\end{frame}

\begin{frame}{Example 1}
(d) \quad   $r = \frac{ed}{1-e\sin\theta}$  \pause    \quad $\longrightarrow \quad r = \frac{d}{1-\sin\theta}$ \newline\\ \pause
    \begin{itemize}
        \item Parabola (opens up or down)  \pause  \newline\\
        \item Vertex at $\left(0, -\frac{1}{2}d\right)$ \pause \newline\\
        \item $d > 0$ opens up    \pause  \newline\\
        \item $d < 0$ opens down   \pause  \newline\\
        \item $x$-intercepts at $\left(\pm d, 0\right)$
    \end{itemize}
\end{frame}

\begin{frame}{Follow-up to Example 1}
Notice each of the previous graphs in Example 1 were parabolas. This is the case when $e = 1$. The directrix lines were either $x = \pm d$ or $y = \pm d$, and the focal diameter is $2d$.
\end{frame}

\begin{frame}{Example 2}
Examine the graphs of each of the following for different values of $d$, but with $0 < e < 1$ and $e > 1$.  \newline\\ 
(a) \quad   $r = \frac{ed}{1+e\cos\theta}$  \newline\\  \pause
For $0 < e < 1$:    \pause  \newline\\

Ellipse (wide)
\end{frame}

\begin{frame}{Example 2}
(a) \quad   $r = \frac{ed}{1+e\cos\theta}$  \newline\\  \pause
For $e > 1$:    \pause  \newline\\

Hyperbola (opening left and right)
\end{frame}


\begin{frame}{Example 2}
Examine the graphs of each of the following for different values of $d$, but with $0 < e < 1$ and $e > 1$.  \newline\\ 
(b) \quad   $r = \frac{ed}{1-e\cos\theta}$  \newline\\  \pause
For $0 < e < 1$:    \pause  \newline\\

Ellipse (wide)
\end{frame}

\begin{frame}{Example 2}
(b) \quad   $r = \frac{ed}{1-e\cos\theta}$  \newline\\  \pause
For $e > 1$:    \pause  \newline\\

Hyperbola (opening left and right)
\end{frame}


\begin{frame}{Example 2}
Examine the graphs of each of the following for different values of $d$, but with $0 < e < 1$ and $e > 1$.  \newline\\ 
(c) \quad   $r = \frac{ed}{1+e\sin\theta}$  \newline\\  \pause
For $0 < e < 1$:    \pause  \newline\\
Ellipse (tall)
\end{frame}

\begin{frame}{Example 2}
(c) \quad   $r = \frac{ed}{1+e\sin\theta}$  \newline\\  \pause
For $e > 1$:    \pause  \newline\\

Hyperbola (opening up and down)
\end{frame}

\begin{frame}{Example 2}
Examine the graphs of each of the following for different values of $d$, but with $0 < e < 1$ and $e > 1$.  \newline\\ 
(d) \quad   $r = \frac{ed}{1-e\sin\theta}$  \newline\\  \pause
For $0 < e < 1$:    \pause  \newline\\

Ellipse (tall)
\end{frame}

\begin{frame}{Example 2}
(d) \quad   $r = \frac{ed}{1-e\sin\theta}$  \newline\\  \pause
For $e > 1$:    \pause  \newline\\

Hyperbola (opening up and down)
\end{frame}

\begin{frame}{Properties From Example 2}
In the previous example, the graphs in which $0 < e < 1$ were ellipses. \\[18pt]  \pause

Major axis length is $\frac{2ed}{1-e^2}$ \\[18pt] \pause
Minor axis length is $\frac{2ed}{\sqrt{1-e^2}}$.   
\end{frame}

\begin{frame}{Properties From Example 2}

If $e > 1$, the graph is a hyperbola.   \\[18pt] \pause

Transverse axis length $\frac{2ed}{e^2-1}$  \\[18pt]    \pause
Conjugate axis length $\frac{2ed}{\sqrt{e^2-1}}$.
\end{frame}

\begin{frame}{Example 3}
Identify the conic for each.    \newline\\
(a) \quad $r = \frac{4}{1-\sin\theta}$  \newline\\  \pause
$e = 1 \longrightarrow$ Parabola (opens up).   \newline\\  \pause
Vertex: $(0, -2)$   \newline\\  \pause
Goes through $(\pm 4, 0)$
\end{frame}

\begin{frame}{Example 3}
(b) \quad $r = \frac{12}{3-\cos\theta}$ \pause  \quad $\longrightarrow \quad r = \frac{4}{1-1/3\cos\theta}$   \\[18pt] \pause
$e = \frac{1}{3} \longrightarrow$ Ellipse (wide) \\[15pt] \pause
$\frac{1}{3}d = 4 \longrightarrow d = 12$   \\[15pt]  \pause
Major axis length: $\frac{2(1/3)(12)}{1-(1/3)^2} = 9$  \\[15pt]   \pause
Minor axis length: $\frac{2(1/3)(12)}{\sqrt{1-(1/3)^2}} = 6\sqrt{3}$
\end{frame}

\begin{frame}{Example 3}
(c) \quad $r = \frac{6}{1+2\sin\theta}$ \newline\\  \pause
$e = 2$: Hyperbola (opens up and down) \newline\\ \pause
$2d = 6 \longrightarrow d = 3$  \newline\\  \pause
Transverse axis length: $\frac{2(2)(3)}{2^2-1} = 4$ \\[15pt]    \pause
Conjugate axis length: $\frac{2(2)(3)}{\sqrt{2^2-1}} = 4\sqrt{3}$
\end{frame}


\begin{frame}{Polar Form of Rotated Conics}
For constants $\ell > 0, \, e \geq 0, \text{ and } \phi,$ the graph of 
\[
r = \frac{\ell}{1-e\cos(\theta-\phi)}
\]
is a conic section with eccentricity $e$ and one focus at $(0,0)$.
\end{frame}

\begin{frame}{Polar Form of Rotated Conics}
    If $e = 0$, the graph is a circle centered at $(0,0)$ with radius $\ell$.
\end{frame}

\begin{frame}{Polar Form of Rotated Conics}
    If $e \neq 0$, the conic has a focus at $(0,0)$ and the directrix contains the point with polar coordinates $(-d,\phi)$ where $d = \frac{\ell}{e}$.
    \begin{itemize}
        \item If $0 < e < 1$, graph is an ellipse with major axis length $\frac{2ed}{1-e^2}$ and minor axis length $\frac{2ed}{\sqrt{1-e^2}}$. \\[15pt] \pause
        \item If $e=1$, graph is a parabola with focal diameter $2d$.   \\[15pt]    \pause
        \item If $e > 1$, graph is a hyperbola with transverse axis length $\frac{2ed}{e^2-1}$ and conjugate axis length $\frac{2ed}{\sqrt{e^2-1}}$
    \end{itemize}
\end{frame}


\end{document}
